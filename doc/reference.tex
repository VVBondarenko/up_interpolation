\documentclass[a4paper,12pt]{article}
\usepackage[T2A]{fontenc}
\usepackage[utf8]{inputenc}
\usepackage[russian,english]{babel}
% \usepackage{gnuplottex}

\usepackage{amsmath}

\begin{document}
  \title{Function Reference}
  \maketitle
  
    
  \newcommand{\func}[3]{
    \section{{#1}}
    \paragraph{Argumets} {#2}
    \paragraph{Function} {#3}
  }
  \func {up\_init}{void}{Initialize up function for further usage.}
  \func{f\_up}{double X}{Returns value of $up(X)$}
  \func{interpolate}{double X, double Values[][4], double step, int N}{Returns interpolated value of function in X. Function defined inside of Values[N][4],
  where Values[][i], $i=0$ is for coordinates, $i=1$ is for values of function, $i=1+k, k>=1$ is for function derivatives. Uses simpliest algorithm with 
  single "crossover" of $up(x)$. Doesn't use the information about derivatives.}
  \func{interpolateCIP}{double X, double Values[][4], double step, int N}{Similar to usual "interpolate", but uses the information about derivatives.}
  \func{interpolateDD}{double X, double Values[][4], double step, int N}{Similar to "interpolate", but uses information about derivatives at boundary points,
  and built using double "crossover" of $up(x)$ in interpolation.}
  \func{interpolateDDd}{double X, double Values[][4], double step, int N}{$(Experimental)$ \space Similar to "interpolateDD", but tries to use information 
  about derivatives not only for boundary parts.}
  
  %     This is the Wikibook about LaTeX
%     supported by {#1} and {#2}!}
  % in the document body:
%   \begin{itemize}
%   \item \wbalTwo{John Doe}
%   \item \wbalTwo[lots of users]{John Doe}
%   \end{itemize}
  
%   \section{init\_up()}
%   \paragraph{Argumets} (void)
%   \paragraph{Function} function up for further usage.
%   
%   \section{f\_up()}
%   \paragraph{Argumets} (double X)
%   \paragraph{Function}
\end{document}
